\documentclass[12pt]{article}
\usepackage[english]{babel}
\usepackage{natbib}
\usepackage{url}
\usepackage[utf8x]{inputenc}
\usepackage{amsmath}
\usepackage{graphicx}
\graphicspath{{images/}}
\usepackage{parskip}
\usepackage{fancyhdr}
\usepackage{vmargin}
\setmarginsrb{3 cm}{2.5 cm}{3 cm}{2.5 cm}{1 cm}{1.5 cm}{1 cm}{1.5 cm}

\title{Agujeros Negros}								% Title
\author{Samuel Macias Ayala}								% Author
\date{\today}											% Date

\makeatletter
\let\thetitle\@title
\let\theauthor\@author
\let\thedate\@date
\makeatother

\pagestyle{fancy}
\fancyhf{}
\rhead{\theauthor}
\lhead{\thetitle}
\cfoot{\thepage}

\begin{document}

%%%%%%%%%%%%%%%%%%%%%%%%%%%%%%%%%%%%%%%%%%%%%%%%%%%%%%%%%%%%%%%%%%%%%%%%%%%%%%%%%%%%%%%%%

\begin{titlepage}
	\centering
    \vspace*{0.5 cm}
    \includegraphics[scale = 0.75]{agujeronegro.jpg}\\[1.0 cm]	% University Logo
	\textsc{\Large Fundamentos de Investigación}\\[0.5 cm]				% Course Code
	\textsc{\large Unidad 3}\\[0.5 cm]				% Course Name
	\rule{\linewidth}{0.2 mm} \\[0.4 cm]
	{ \huge \bfseries \thetitle}\\
	\rule{\linewidth}{0.2 mm} \\[1.5 cm]
	
	\begin{minipage}{0.4\textwidth}
		\begin{flushleft} \large
			\emph{Autor:}\\
			Samuel Macias Ayala
			\end{flushleft}
			\end{minipage}~
			\begin{minipage}{0.4\textwidth}
			\begin{flushright} \large
			\emph{Número de Control:} \\
			15211316									% Your Student Number
		\end{flushright}
	\end{minipage}\\[2 cm]
	
	{\large  \thedate}\\[2 cm]
 
	\vfill
	
\end{titlepage}

%%%%%%%%%%%%%%%%%%%%%%%%%%%%%%%%%%%%%%%%%%%%%%%%%%%%%%%%%%%%%%%%%%%%%%%%%%%%%%%%%%%%%%%%%

\tableofcontents
\pagebreak

%%%%%%%%%%%%%%%%%%%%%%%%%%%%%%%%%%%%%%%%%%%%%%%%%%%%%%%%%%%%%%%%%%%%%%%%%%%%%%%%%%%%%%%%%

\section{Resumen}
Los agujeros negros son un lugar o una región del universo en la cual la fuerza de la gravedad es tan, pero tan fuerte que ningún tipo de material ni partícula(ni siquiera la luz)alguna puede escapar de allí.\\\\
Se forman cuando una estrella de gran tamaño muere y esta se contrae tanto que no soporta su mismo campo de gravedad, y entonces es como si la misma estrella hubiera caído en un agujero infinitamente hondo y no cesase nunca de caer. Y como ni siquiera la luz puede escapar, el objeto comprimido será negro. Literalmente, un agujero negro.
\\\\
Se dice que en el centro de cualquier galaxia existe un agujero negro supermasivo que es el que le da forma y la hace girar. Estos poseen una masa de miles de millones de la masa de nuestro sol.
\\\\
Un agujero negro es mucho más que un simple hueco en el espacio. En su interior las propiedades del espacio y del tiempo se alteran de maneras insólitas. La frontera del agujero negro está marcada por su radio gravitacional, también conocido como el horizonte de eventos.\\\\
Las propiedades del espacio-tiempo en el interior de un agujero negro son tan extrañas, que todavía hoy en día hay quien trata de demostrar a toda costa que unos objetos tan absurdos no pueden existir. Su realidad, sin embargo, es inevitable si la teoría de la relatividad es correcta.\\\\
\newpage

\section{Introducción}
En la siguiente investigación profundizaremos sobre lo que es un gran misterio en el espacio, los agujeros negros, que son un fenómeno natural del universo. Si bien esta comprobado que existen aún quedan muchas incognitas sobre que es lo que hay dentro de estos.\\ Pero si eres alguien que quiere aprender sobre lo que es un agujero negro entonces te servira de mucho esta investigación. Se explicará lo que es, como se forma y que efecto causa a su alrededor entre otras cosas. Espero que con esto se aclaren tus dudas sobre los agujeros negros.
\newpage

\section{Justificación}
Siempre he sido alguien que le ha gustado la astronomía, me encantan los documentales que pasan por la tele hablando del universo. Por eso mismo elegi a los agujeros negros ya son algo muy fascinante desde mi punto de vista.\\
El motivo por el cual quise hacer mi investigación sobre los agujeros negros es por que siempre me ha causado intriga estos objetos cosmicos. Siempre tuve dudas sobre lo que en realidad eran, de como se formaban y si en verdad existian.\\

\newpage

\section{Objetivos generales}
El objetivo general de esta investigación es el de aclarar cualquier duda que tenga el lector sobre los agujeros negros. Además de resolver mitos que la gente cree de estos fenómenos del universo.
\subsection{Objetivos Específicos}
\begin{enumerate}
	\item Aprender lo que es un agujero negro.
	\item Conocer como se forma un agujero negro.
	\item Saber como esta conformado un agujero negro.
	\item Explicar que pasaria dentro de un agujero negro.
	\item Pasar la materia de Fundamentos de Investigación.
\end{enumerate}
\newpage

\section{Agujeros Negros}
\subsection{Historia}
El término “Black Hole” fue utilizado por primera vez por una periodista, Ann Ewing, en 1964 al realizar un informe de una reunión de la Asociación Americana de Ciencia Avanzada (American Association for the Advancement of Science) y escribir un artículo en Science News Letter titulado “Black Holes in Space”. Unos días más tarde, Albert Rosenfeld publica en la revista LIFE que el colapso gravitacional de una estrella puede terminar en un agujero negro invisible en el Universo. Sin embargo,  es el físico teórico estadounidense, John Archibald Wheeler, el que populariza el término en una conferencia impartida en 1967 en la Universidad de Columbia, fija el término de “black hole - agujero negro” al referirse a lo que se venía denominando estrella en colapso gravitatorio continuo.
\\\\
Más allá del nombre, el concepto aparece mucho antes. El filósofo inglés John Michell, en 1783, casi 100 años después de que Newton escribiera su ley de la gravitación universal, en un informe a la Royal Society de Londres, menciona la idea de que un cuerpo masivo no dejaría escapar la luz.  Unos años más tarde, en 1796, el matemático Pierre-Simon Laplace predijo su existencia en el libro Exposition du système du Monde, donde habla de estrellas negras.\\\\
Cuando en la primera década del siglo XX (1905 para ser más exactos) Albert Einstein publicó la teoría de la relatividad muy pocos pudieron visualizar el gran impacto que ésta teoría podría tener en la física y en el entendimiento de los fenómenos estelares. Con la observación de un eclipse solar en 1919 se corroboró que su teoría tenía grandes bases para poder entender mejor al universo. Si bien Einstein no recibió por éste trabajo el premio Nóbel de física al menos brindó a los astrónomos la posibilidad de poder entender los descubrimientos que se realizarían en las décadas posteriores. Uno de éstos descubrimientos fue la existencia de los agujeros negros.\\\\
Los agujeros negros, vistos desde la perspectiva que nos brinda la teoría de la relatividad y de las teorías que de ella se derivaron nos muestran una inquietante visión de un universo que día a día nos sorprende más, con estrellas evolucionando, planetas que podrían albergar vida y un misterioso comportamiento en el interior de los agujeros negros en donde las cosas no pueden ser explicadas con los conocimientos que poseemos, pues allí dentro, ni la física ni las matemáticas que conocemos (o que estamos conociendo) se cumplen.\\\\
El sólo hecho de saber que las cosas tal como las conocemos no funcionan siguiendo nuestra lógica convierte de por sí a los agujeros negros en un fenómeno más que interesante. ¿Te puedes imaginar poder tener un movimiento cuya distancia no puede ser medida? ¿O tal vez imaginar un disco compacto con cinco caras y que pueda ser a la vez bidimensional?. Cosas tan extrañas como las que han sido mencionadas son las que provocan el interés en los agujeros negros.\\\\
¿Qué pasará con los agujeros negros en el universo?, ¿cómo se comportan y qué tamaño tienen?, ¿un agujero negro acabará con la existencia del universo tal como lo conocemos? éstas preguntas frecuentes e inquietantes intentarán ser resueltas en los vínculos siguientes y tratarán de mostrarte de manera simple lo que hasta ahora conocemos acerca de los agujeros negros.[1]
\\%Link Azul Marino[1]

\subsection{¿Qué son?}
Según la NASA, es un agujero negro, digamos que se trata de un lugar o una región del universo en la cual la fuerza de la gravedad es tan, pero tan fuerte que ningún tipo de material ni partícula alguna (ni siquiera de luz y por eso su nombre) puede escapar de allí. La enorme fuerza gravitacional que allí existe, como resultado de una fuerte concentración de masa, se hace tan fuerte porque esta última se comprime en un espacio muy pequeño. Como ni siquiera la luz puede salir de allí, los agujeros negros no pueden visualizarse, aunque actualmente, con nuestros avances tecnológicos y especialmente con nuestros más modernos telescopios espaciales, es posible ubicarlos.
 \\\\
 Un agujero negro puede ser muy grande o sorprendentemente pequeño, pudiendo existir incluso algunos del tamaño de un átomo. Lo más fascinante es que, a pesar de tener un tamaño tan pequeño como el de un átomo, pueden contener la masa de una de las montañas más grandes de nuestro planeta dentro, mientras que uno grande puede tener el tamaño de 20 soles.[2]
 \\%Link Amarillo[2]
 \\
 Los agujeros negros son cuerpos celestes con un campo gravitatorio tan fuerte que ni siquiera la radiación electromagnética (La luz) puede escapar de su proximidad cayendo inexorablemente en el agujero.\\\\
 Un campo de estas características puede corresponder a un cuerpo de alta densidad con una masa relativamente pequeña, como la del Sol o inferior, que está condensada en un volumen mucho menor, o a un cuerpo de baja densidad con una masa muy grande, como una colección de millones de estrellas en el centro de una galaxia.[3]\\%Link Naranja[3]
 
\subsubsection{Horizonte de eventos}
El cuerpo está rodeado por una frontera esférica, llamada "horizonte de sucesos", a través de la cual la luz puede entrar, pero no puede salir, por lo que parece ser completamente negro.\\\\
Se llama Horizonte de sucesos ya que el único suceso que puede ocurrir una vez pasada la frontera es el de seguir cayendo en el agujero, ya que no hay velocidad posible suficientemente grande como para escapar de la atracción gravitatoria, ni siquiera a la velocidad de la luz se puede escapar (Aproximadamente 300.000 kilómetros por segundo).[3]\\\\%Link Naranja
El horizonte posee algunas propiedades geométricas bastante raras. Para alguien que esté observando a alguna distancia del agujero negro, el horizonte de eventos parece una superficie esférica muy tranquila y estática. Pero cuando se aproxima del horizonte puede percibir que él se mueve a una velocidad espantosa.
\\\\
El horizonte es estático en un sentido pero en otro él mismo se está moviendo a la velocidad de la luz.[4]
\\%Link Gris[4]
\subsubsection{Singularidad}
De acuerdo a la teoría general de la relatividad, una singularidad es un punto teórico con volumen cero y densidad infinita. Es el resultado al que cualquier masa que se convierte en agujero negro tiene que colapsar.
\\\\
Las singularidades se forman cuando se produce colapso gravitacional de estrellas masivas. El Big Bang debió ser una singularidad. Una singularidad es un lugar en el que la densidad de materia y la curvatura del espacio se hacen infinitas, y no tiene significado desde el punto de vista físico teórico.
\\\\
Según la hipótesis de la censura cósmica, propuesta por el físico y matemático británico Roger Penrose, cuando se forman ese tipo de singularidades, éstas no se encuentran desnudas, en el sentido de ser visibles a observadores externos, sino que están escondidas discretamente en el interior del horizonte de un agujero negro y, por tanto, son aceptables.
\\\\
Otro físico británico, Stephen William Hawking, apoya la teoría de que la creación del Universo tuvo su origen a partir de una Gran Explosión o Big Bang, surgida de una singularidad o un punto de distorsión infinita del espacio y el tiempo.[5]
\\\\%Link Rosa[5]
\subsection{¿Como se forman?}
Para entender lo que es un agujero negro empecemos por una estrella como el Sol, que tiene un diámetro de 1.390.000 kilómetros y una masa 330.000 veces superior a la de la Tierra.\\\\
Teniendo en cuenta esa masa y la distancia de la superficie al centro se demuestra que cualquier objeto colocado sobre la superficie del Sol estaría sometido a una atracción gravitatoria unas 28 veces superior a la gravedad terrestre en la superficie del planeta. Una estrella corriente conserva su tamaño normal gracias al equilibrio entre una altísima temperatura central, que tiende a expandir la sustancia estelar, y la gigantesca atracción gravitatoria, que tiende a contraerla y estrujarla.
\\\\
Si en un momento dado la temperatura interna desciende, la gravitación se hará dueña de la situación. La estrella comienza a contraerse y a lo largo de ese proceso la estructura atómica del interior se desintegra. En lugar de átomos habrá ahora electrones, protones y neutrones sueltos. La estrella sigue contrayéndose hasta el momento en que la repulsión mutua de los electrones contrarresta cualquier contracción ulterior.\\\\
La estrella es ahora una «enana blanca». Si una estrella como el Sol sufriera este colapso que conduce al estado de enana blanca, toda su masa quedaría reducida a una esfera de unos 16.000 kilómetros de diámetro, y su gravedad superficial (con la misma masa pero a una distancia mucho menor del centro) sería 210.000 veces superior a la de la Tierra.\\\\
En determinadas condiciones la atracción gravitatoria se hace demasiado fuerte para ser contrarrestada por la repulsión electrónica. La estrella se contrae de nuevo, obligando a los electrones y protones a combinarse para formar neutrones y forzando también a estos últimos a apelotonarse en estrecho contacto. La estructura neutrónica contrarresta entonces cualquier ulterior contracción y lo que tenemos es una «estrella de neutrones», que podría albergar toda la masa de nuestro sol en una esfera de sólo 16 kilómetros de diámetro. La gravedad superficial sería 210.000.000.000 veces superior a la que tenemos en la Tierra.
\\\\
Según la teoría de la relatividad, la luz emitida por una estrella pierde algo de su energía al avanzar contra el campo gravitatorio de la estrella. Cuanto más intenso es el campo, tanto mayor es la pérdida de energía, lo cual ha sido comprobado experimentalmente en el espacio y en el laboratorio.\\\\
La luz emitida por una estrella ordinaria como el Sol pierde muy poca energía. La emitida por una enana blanca, algo más; y la emitida por una estrella de neutrones aún más. A lo largo del proceso de colapso de la estrella de neutrones llega un momento en que la luz que emana de la superficie pierde toda su energía y no puede escapar.
\\\\
Un objeto sometido a una compresión mayor que la de las estrellas de neutrones tendría un campo gravitatorio tan intenso, que cualquier cosa que se aproximara a él quedaría atrapada y no podría volver a salir. Es como si el objeto atrapado hubiera caído en un agujero infinitamente hondo y no cesase nunca de caer. Y como ni siquiera la luz puede escapar, el objeto comprimido será negro. Literalmente, un «agujero negro».[6]
\\%Link Rojo[6]
\subsubsection{Formación de estrellas - El límite de Chandrasekhar}
Para empezar, no todas las estrellas se pueden convertir en agujeros negros, para ello deben de cumplir ciertos requisitos como por ejemplo el tamaño, tiempo de vida, entre otras características.
\\\\
Las estrellas se forman a partir de grandes concentraciones de gas, principalmente hidrógeno, por efectos gravitatorios los átomos que conforman estos gases empezarán a colapsar unos contra otros contrayéndose y generando un calentamiento del gas, el calor poco a poco se incrementará llegando a generarse reacciones importantes entre los átomos (transformación de moléculas de Hidrógeno en Helio). Estas reacciones provocan emanaciones de energía altísimas que le dan a las estrellas la luminosidad característica. Todo esto ocurre hasta un momento en que los átomos llegan a alcanzar un equilibrio a partir del cual dejan de contraerse. El Sol se encuentra en estos momentos en este equilibrio, en el que no existe ningún tipo de contracción por parte de sus componentes.\\\\
Ahora bien, durante el período de tiempo que toma el proceso de contracción de los átomos la estrella sigue acumulando más gases y crece en tamaño, este tamaño fue estudiado por Subrahmanyan Chandrasekhar, quien indicó el tamaño máximo que una estrella puede alcanzar antes de llegar a consumir todo su combustible natural. Chandrasekhar descubrió el límite al cual una estrella puede crecer de manera que su masa pueda llegar a ser tal que la estrella llegue al límite de soporte de su gravedad. ¿Qué significa lo anterior? que si la estrella es muy grande su gravedad podría provocar que esta "se derrumbe sobre sí misma" (para entenderlo piensa en un huevo cayendo a 400 metros de profundidad bajo el mar, lo que sucedería es que el huevo se rompería por efecto de la presión del agua la cual se ejerce de manera perpendicular sobre la superficie del huevo antes de caer al fondo del mar).
\\\\
Bueno, sucede entonces que este señor Chandrasekhar calculó matemáticamente que la masa crítica de una estrella sería igual a 1,5 veces la masa del sol a ésta masa se le denomina el límite de Chandrasekhar, por debajo de éste límite encontramos a las enanas blancas y las estrellas de neutrones mientras que por encima de ese límite... bueno no fue hasta 1939 que se logró explicar que sucedería con una estrella con una masa mayor a la del límite de Chandrasekhar, esa estrella poseería un campo gravitatorio tan fuerte que los rayos de luz emanados de la estrella empiezan a irradiarse hacia la superficie (como un boomerang), poco a poco los rayos de luz se inclinan con mayor fuerza hacia la misma estrella de la cual emanan. A lo lejos un observador contemplará como la estrella pierde luminosidad tornándose roja (un efecto parecido a cuando las baterías de una lámpara se van acabando de a pocos), Cuando la estrella llegue a alcanzar un radio crítico el campo gravitatorio crecerá de manera exponencial llegando finalmente a atrapar a la misma luz dentro de ella.[1]
\\\\%Link Azul Marino
\subsection{Agujeros Negros Supermasivos}
Son los que existen en el centro de las galaxias y hacen girar a éstas, poseen una masa de miles de millones de la masa de nuestro sol.\\\\
La práctica totalidad de la comunidad científica, en el campo de la astronomía, se ha puesto de acuerdo en que el núcleo de cada galaxia contiene un agujero negro muy masivo o supermasivo. Esta idea se planteó hace muchos años y se ha reforzado debido a los múltiples descubrimientos que han localizado agujeros negros prácticamente en cada una de las galaxias que se estudia.\\\\
Cada galaxia puede contener decenas de agujeros negros, pero solo uno parece ser el responsable de mantener unidas a millones, cientos de miles de millones o billones de estrellas dentro de una galaxia. Este supermasivo agujero negro se localiza en los núcleos de estas galaxias, esencialmente en las galaxias espirales y elípticas, ya que otras galaxias irregulares, de anillo, etc, no son perdurables en el tiempo (no obstante es posible que se conviertan en espirales o elípticas).\\\\
Un agujero negro es un objeto peculiar, diríase que exótico, único y de momento poco comprensible, aunque las bases sobre ellos están establecidas. Un agujero negro supermasivo, con una masa de millones o miles de millones de veces la solar, es en esencia un túnel enorme en el espacio, con una gravedad inimaginable que atrae a billones de estrellas, que giran a su alrededor durante miles de millones de años, excepto las más cercanas, que son absorbidas y desintegradas.
Uno de los supermasivos agujeros negros de mayor masa que se conocen se localiza en la galaxia elíptica gigante NGC 4889, ¡con una masa de 21.000 millones de soles! Esta galaxia de 300.000 años luz de extensión (tres veces mayor que la nuestra), está enclavada en el centro de un subcúmulo de galaxias, una parte del cúmulo de galaxias de Coma, y está engullendo otras galaxias mayores, además de atraer a otro subcúmulo de galaxias de Coma. La masa del agujero negro de esta galaxia es una quinta parte del total de la masa de la Vía Láctea.[7]
\\\\%Link Negro[7]
Mirando muy muy lejos se ven las galaxias tal como eran cuando el universo era mucho más joven, hasta la décima parte de su edad actual. Estas galaxias, llamadas quásares, emiten casi toda su radiación desde el mismo centro, como volcanes en erupción. La única explicación aceptada de su funcionamiento es un agujero negro gigante que se alimenta violentamente, engullendo a borbotones y emitiendo ‘salpicaduras’ que vemos como la radiación del quásar. Pero si las galaxias tenían agujeros negros gigantes en su adolescencia, deben mantenerlos en su vejez, ocultos como dragones dormidos…
\\\\
Justo en el centro de nuestra galaxia, a solo 27.000 años luz de nosotros, hay un anillo de gases que emite radiación mientras gira a grandísimas velocidades. Fue descubierto en 1974 por los estadounidenses Bruce Balick y Robert Brown, que lo denominaronSagittarius A*. Con el tiempo se ha comprobado que las estrellas en su vecindad trazan órbitas rapidísimas. Para que se produzcan estos fenómenos es necesario que haya en el centro galácticouna gigantesca masa de tres o cuatro millones de soles, que, según muchos científicos, sólo puede ser un agujero negro.\\\\
La hipótesis de que todas las grandes galaxias tienen un agujero negro gigante en el centro, testigo mudo de su pasado quásar, ha dejado hace tiempo de ser un ámbito limitado a los ‘pioneros’ de los años 60, como el ruso Yakov Zeldovich, el norteamericano Edwin Salpeter, o los británicos Donald Lynden-Bell y Martin Rees. Hoy, el papel de estos ‘monstruos’ en la vida de las galaxias es una de las áreas de trabajo esenciales en la astrofísica contemporánea.\\\\
Nuestro ‘dragón’ particular en el centro de la Vía Láctea pesa lo equivalente a cuatro millones de soles. Sin embargo, su tamaño sería relativamente pequeño en el club de los monstruos. Se estima que muchos quásares tienen agujeros negros con masas de miles de millones de soles. Incluso las estimaciones realizadas para la zona central de la galaxia Andrómeda revelan que su agujero negro podría ser unas 100 veces mayor que el nuestro. Dado que ambas galaxias están en rumbo de colisión parece claro quién llevará la voz cantante en el baile cósmico que tendrá lugar en unos 4.000 millones de años.[11]
\\\\%Link11

\subsection{Espacio-Tiempo}
Antes de seguir adelante me gustaría disipar un error conceptual muy común acerca de los agujeros negros: pese a lo que hemos visto en incontables películas, los agujeros negros no son de ninguna manera aspiradoras cósmicas que se tragan todo lo que se les acerca. Si en este momento el Sol se convirtiera en un agujero negro (es decir, si se comprimiera hasta alcanzar su radio gravitacional), no notaríamos ningún cambio (fuera de que nos daría mucho frío y estaría muy oscuro). La Tierra seguiría en su órbita tan campante, sin alterarse. Los agujeros negros sólo resultan peligrosos si uno se aproxima mucho, a distancias cercanas al radio gravitacional. En el caso del Sol tendrías que acercarte a unos tres kilómetros, pero no a tres kilómetros de la superficie actual del Sol, sino a tres kilómetros del centro del Sol si toda su masa estuviera concentrada en un punto.\\\\
Un agujero negro es mucho más que un simple hueco en el espacio. En su interior las propiedades del espacio y del tiempo se alteran de maneras insólitas. La frontera del agujero negro está marcada por su radio gravitacional, también conocido como el horizonte de eventos.\\\\
Las propiedades del espacio-tiempo en el interior de un agujero negro son tan extrañas, que todavía hoy en día hay quien trata de demostrar a toda costa que unos objetos tan absurdos no pueden existir. Su realidad, sin embargo, es inevitable si la teoría de la relatividad es correcta.\\\\
Aún así, es un hecho que hasta ahora no se ha detectado ningún agujero negro de manera directa. Las pruebas indirectas, por otro lado, se acumulan día a día. Cada vez se descubren más regiones del espacio donde se encuentra una enorme cantidad de materia (que se puede detectar por su influencia gravitacional sobre el movimiento de los cuerpos cercanos) concentrada en un volumen tan pequeño, que debe tener un radio menor que su radio gravitacional. La física de hoy no admite otra interpretación de estas regiones: tienen que ser agujeros negros. Por eso hoy en día se cree, por ejemplo, que hay agujeros negros gigantescos en el centro de casi todas las galaxias, incluyendo a la nuestra.
\\\\
Las propiedades extrañas de los agujeros negros no se limitan a la existencia del horizonte de eventos, la mezcla entre espacio y tiempo y la inevitable caída a la singularidad. Ya desde la primera mitad del siglo XX se había descubierto que en el interior del agujero negro debe existir no sólo una singularidad de campos gravitacionales infinitos, sino también un túnel que llevaría, de haberlo, a otro universo. Este túnel se conoce en lenguaje científi co como puente de Einstein- Rosen en honor a Albert Einstein y Nathan Rosen, los científi cos que dedujeron su existencia por primera vez. En el lenguaje más popular también se le conoce como "agujero de gusano".\\\\
El agujero de gusano que se supone se encuentra en el interior de un agujero negro sería un puente entre dos universos exteriores distintos (el nuestro y algún otro, digamos), en cada uno de los cuales habría un agujero negro y un horizonte. Sin embargo, este túnel no se puede usar para viajar a otros posibles universos. El túnel aparece y desaparece sólo una vez, y lo hace tan rápido que incluso viajando a la velocidad de la luz (y no se puede ir más rápido) el túnel se cerraría antes de que pudieras atravesarlo. La singularidad del agujero negro puede entenderse también como el resultado del cerrarse del túnel: al derrumbarse éste sobre nosotros, nos veríamos de pronto atrapados en una región del espacio-tiempo que desaparece.\\\\
Es posible imaginarse agujeros de gusano que no se cierren y que nos permitan llegar a otros universos o a regiones lejanas de nuestro propio Universo en un abrir y cerrar de ojos, pero esa es una historia para otro momento.[8]
\\%Link Verde[8]

\subsection{¿Qué hay dentro de un agujero negro?}
En el centro de un agujero negro hay lo que los físicos llaman “singularidad”, o un punto donde cantidades extraordinariamente grandes de material se compactan en una cantidad infinitamente pequeña de espacio.
“Desde un punto de vista teórico, la singularidad es algo que se vuelve infinitamente grande”, dijo la física Sabine Hossenfelder del Instituto Nórdico de Física Teórica.\\\\
Técnicamente, ese “algo” es la curvatura del espacio, o la gravedad extrema que los científicos han observado en la presencia de enormes masas como las de grandes planetas y estrellas.\\\\
Similar a como una lámina de goma estirada se hunde alrededor de una bola de bolos, los objetos masivos pueden causar que el espacio-tiempo se curve alrededor de ellos. Y mientras más masivo sea el objeto, más pronunciada será la curvatura. Teorizado por primera vez por Einstein, no existe un efecto más extremo que el de un agujero negro, cuyo centro representa una curva de curvatura infinita. Como un agujero sin fondo en una lámina de goma, la fuerza se vuelve infinitamente mayor cuanto más se adentre en el interior del agujero.\\\\
Alrededor de la singularidad, las partículas y materiales son comprimidos. Cuando la materia colapsa en un agujero negro, su densidad se vuelve infinita porque debe caber dentro de un punto que, según las ecuaciones, es tan pequeño que no tiene dimensión.
Algunos científicos han discutido si las ecuaciones teóricas que describen los agujeros negros son correctas, es decir, si realmente existen.
\\\\
Nadie puede estar seguro de que su singularidad no describe una realidad física, dijo Hossfelder. Sin embargo, la mayor parte de los físicos diría que la singularidad, teorizada por las ecuaciones, en realidad no existe. Si la singularidad fuese “realmente real”, significaría que “la densidad de la energía fuese infinitamente alta en un punto”, exactamente el centro del agujero negro, dijo.
\\\\
No obstante, nadie puede estar seguro, debido a que no existe una teoría cuántica de gravedad completa y los interiores de los agujeros negros son imposibles de observar.[9]
\\%Link Azul Cielo[9]

\subsection{Agujeros Blancos}
Una estrella cuya masa sea superior dos veces a la del Sol no puede acabar en enana blanca o en estrella de neutrones; en algunos casos estallará y arrojará materia suficiente para que su masa llegue a ser inferior al límite, pero no siempre sucederá así. Algunas estrellas se volverán tan pequeñas que sus campos gravitatorios curvaran la luz hasta el punto de que ésta retorne hacia la estrella. Ni la luz ni ninguna otra cosa podrá escapar de allí. Esas estrellas se convertirán en agujeros negros. 
\\\\Las leyes de la física son simétricas en el tiempo; en consecuencia, si existen objetos llamados agujeros negros donde caen cosas que no pueden salir, ha de haber otros objetos de donde las cosas puedan salir pero no caer. Cabria denominarlos agujeros blancos.\\\\
Si se analizan en detalle las ecuaciones de las que se derivan las propiedades relativistas vamos a encontrar siempre que teóricamente existe una solución simétrica para cada una de ellas, es decir, así como tenemos la idea de que para la materia existe la antimateria, o a lo negro se opone lo blanco, de igual manera podemos deducir teóricamente que debe de existir algo que posea características completamente opuestas a la de los agujeros negros.
\\\\
Para este caso, sabemos que los agujeros negros son definidos como un horizonte de sucesos dentro del cual todo objeto no importando su estado es atrapado indefectiblemente por una fuerza gravitatoria inmensa (casi infinita), por oposición podemos entender que debe de existir (al menos teóricamente) un agujero blanco con un horizonte de sucesos en donde todo lo que esté dentro de él será violentamente repelido, tal vez con una fuerza inmensa (casi infinita) esto nos lleva a pensar en las ideas (nuevamente las cito) de la materia y la antimateria. 
\\\\
Pero lo interesante está en que si bien las matemáticas efectivamente pueden darnos una respuesta simétrica tan controversial, también es cierto que un horizonte de sucesos con esas características es improbable y hasta el momento no ha habido descubrimiento que contradiga su no existencia real.[1]
\\\\%Link Azul Marino


 Se podría imaginar la posibilidad de saltar a un agujero negro en un lugar para salir en otro por un agujero blanco; seria el método ideal de viaje espacial a larga distancia antes mencionado. Todo lo que se necesitaría sería hallar cerca un agujero negro. Tal forma de viaje pareció factible en un primer momento. Hay soluciones de la teoría general de la relatividad de Einstein en las que se puede caer en un agujero negro y salir por un agujero blanco, pero investigaciones ulteriores mostraron que estas soluciones eran muy inestables; una mínima perturbación, como la presencia de una nave espacial, destruiría la "gatera” o conducto desde el agujero negro al blanco. La nave espacial quedaría destrozada por fuerzas de una magnitud infinita. Sería como cruzar en un barril las cataratas del Niágara.[10]
\\%Link Morado[10]
\newpage
\section{Conclusión}
Bueno con todo lo anterior visto podemos concluir en que los agujeros negros son algo muy fascinante en el universo y que hoy en dia siguen siendo un misterio, en unos años más se podría conocer mejor como están conformados. Pero por mientras podemos resolver nuestras dudas más básicas con esta investigación. En lo personal a mi me aclaro muchas preguntas, coomo por ejemplo, yo pensaba que un agujero negro era lo mismo a un agujero de gusano pero indagando me di cuenta que son cosas muy distintas pero a la vez ligadas, aunque estos últimos sean soló hipoteticos. Además descubri cosas que no me esperaba como lo son los agujeros blancos, nunca habia escuchado ni leido de ellos hasta que empece con la investigación. En fin como conclusión final los agujeros negros son uno de los mayores misterios del universo. 
\newpage
\section{Trabajo a futuro}
Si bien esta investigación no le sirve de nada a la ciencia ya que solo es una colección de datos que ya se saben, si puede ser usada por alguien que desconozca del tema y quiera aventurarse y aprender más sobre los agujeros negros.
\newpage
\section{Referencias}
\begin{enumerate}
	\item http://www.cosmopediaonline.com/an-index.html
	\item
	http://www.batanga.com/curiosidades/4143/diferencias-entre-agujeros-negros-y-agujeros-de-gusano
	\item
	http://www.todoelsistemasolar.com.ar/agujeronegro.html
	\item
	http://fisica.laguia2000.com/fisica-cuantica/horizonte-de-eventos-en-un-agujero-negro-espacial
	\item
	http://www.astromia.com/glosario/singularidad.html
	\item
	http://www.astromia.com/astronomia/negroagujero.htm
	\item
	http://www.abc.es/ciencia/20140511/abci-agujeros-negros-supermasivos-tunel-201405081635.html
	\item
	http://www.comoves.unam.mx/numeros/articulo/44/los-agujeros-negros
	\item
	http://www.cosmonoticias.org/que-hay-en-el-centro-de-un-agujero-negro/
	\item
	https://cdn.preterhuman.net/texts/science-and-technology/Stephen20Hawking20-20Agujeros20Negros20y20Pequenos20Universos.pdf 
	\item
	http://blogs.20minutos.es/ciencia-para-llevar-csic/tag/horizonte-de-sucesos/
\end{enumerate}
\end{document}